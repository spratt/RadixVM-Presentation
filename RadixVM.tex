%% CS854 RadixVM Presentation
%% Winter 2016

%%% BEGIN PREAMBLE
\documentclass{beamer}

%% Smart underlining -- from cdi-macros.tex
\def\ul#1{$\underline{\smash{\hbox{#1}}}$}

% TikZ packages (for graphs)
%\usepackage{tkz-graph}
%\usepackage{tkz-berge}
%\usetikzlibrary{snakes}

%% Shortcuts

\newcommand{\bi}{\begin{itemize}}
\newcommand{\ei}{\end{itemize}}

\newcommand{\bn}{\begin{enumerate}}
\newcommand{\en}{\end{enumerate}}

\newcommand{\bd}{\begin{description}}
\newcommand{\ed}{\end{description}}

%%\newcommand{\bf}{\begin{frame}}
%%\newcommand{\ef}{\end{frame}}

%%\title{Presentation Title}
%%\author{Author Name}
%%\institute{Institute, Department, etc.}
%%\date{} % delete this line to display the current date

\mode<presentation>
{
  \usetheme{Madrid}
  \usecolortheme{beaver}
  % or ...

  %\setbeamercovered{transparent}
  % or whatever (possibly just delete it)
}


\usepackage[english]{babel}
% or whatever

%%\usepackage[latin1]{inputenc}
% or whatever

\usepackage{times}
\usepackage[T1]{fontenc}
% Or whatever. Note that the encoding and the font should match. If T1
% does not look nice, try deleting the line with the fontenc.

\title{RadixVM}
% (optional, use only with long paper titles) {Academic Speaking Skills}

\subtitle
{Scalable address spaces for multithreaded applications}
%%{Include Only If Paper Has a Subtitle}

\author[Presented by Simon Pratt]{Austin T. Clements,\\M. Frans Kaashoek,\\Nickolai Zeldovich\\
  \vspace{2em}Presented by Simon Pratt}
% - Give the names in the same order as the appear in the paper.
% - Use the \inst{?} command only if the authors have different
%   affiliation.

%\institute{Cheriton School of Computer Science\\ 
%Department of English\\ Games Institute \\[0.3cm] University of Waterloo} % (optional, but mostly needed)
% - Use the \inst command only if there are several affiliations.
% - Keep it simple, no one is interested in your street address.

\date{February 12, 2016} % (optional, should be abbreviation of conference name)
% - Either use conference name or its abbreviation.
% - Not really informative to the audience, more for people (including
%   yourself) who are reading the slides online

%%\subject{Theoretical Computer Science}
% This is only inserted into the PDF information catalog. Can be left
% out. 

% If you have a file called "university-logo-filename.xxx", where xxx
% is a graphic format that can be processed by latex or pdflatex,
% resp., then you can add a logo as follows:

%%\pgfdeclareimage[width=1.0cm]{logo}{uwlogo}
%%\logo{\pgfuseimage{logo}}

% Delete this, if you do not want the table of contents to pop up at
% the beginning of each subsection:
\AtBeginSubsection[]
{
  \begin{frame}<beamer>{Outline}
   \tableofcontents[currentsection,currentsubsection]
\end{frame}
}


% If you wish to uncover everything in a step-wise fashion, uncomment
% the following command: 
%%\beamerdefaultoverlayspecification{<+->}


%\logo{\includegraphics[scale=0.45]{WaterlooLogo.jpg}}
%%\usecolortheme{crane}

%%% BEGIN DOCUMENT
\begin{document}

\frame[plain]{\titlepage}

\newpage

\begin{frame}{Outline}
  \tableofcontents[pausesections]
  % You might wish to add the option [pausesections]
\end{frame}


% Structuring a talk is a difficult task and the following structure
% may not be suitable. Here are some rules that apply for this
% solution: 

% - Exactly two or three sections (other than the summary).
% - At *most* three subsections per section.
% - Talk about 30s to 2min per frame. So there should be between about
%   15 and 30 frames, all told.

% - A conference audience is likely to know very little of what you
%   are going to talk about. So *simplify*!
% - In a 20min talk, getting the main ideas across is hard
%   enough. Leave out details, even if it means being less precise than
%   you think necessary.
% - If you omit details that are vital to the proof/implementation,
%   just say so once. Everybody will be happy with that.

\section{Basics of Oral Presentations} 

\begin{frame}{Why do good speaking skills matter?}
\begin{itemize}
\item A speech that ignited a generation.
\item A speech that roused a country.
\item Unfortunately the kind of bad speaking you've too often heard!
\end{itemize}
\end{frame}

\section{Preparing the Presentation}

\begin{frame}{Knowing your audience}
\bi
\item Why your audience is there.
\item Background of your audience. Experts or general?
\item What you want to convey to your audience.
\ei
\end{frame}

\section{Evaluating a Presentation}
\begin{frame}{An exercise: Evaluating a talk}
\bi
\item A sample talk will be given.
\item You will evaluate the talk in terms of:
  \bi
  \item Presenter manner.
  \item Structure of talk.
  \item Style and content of slides.
  \item Vocal delivery.
  \item Overall effectiveness.
  \ei
\ei
\end{frame}

\begin{frame}{A final word: The power of good speaking skills}
\bi
\item
The speech that made a President.
\ei
\end{frame}

\end{document}



